
\keywords{%
  программный комплекс,
  веб-интерфейс,
  RTKLIB,
  высокоточное позиционирование
}

\abstractcontent{
Цель данной работы - создание нового интерфейса для пакета программ высокоточного позиционирования RTKLIB. Новый интерфейс позволит наблюдать за статусом RTKLIB и изменять настройки через веб-страницу.

В ходе работы были определены проблемы текущего интерфейса, ограничивающие использование данного проекта. Также были рассмотрены примеры веб-интерфейсов, использующихся в устройствах без других возможностей взаимодействия с пользователем. На основе полученных данных был разработан прототип интерфейса, позволяющего работать с основными функциями RTKLIB через веб-страницу. Для реализации прототипа были спроектированы программные модули, позволяющие автоматизировать работу RTKLIB и передавать данные на веб-страницу для визуализации.

В результате работы, было разработано приложение, представляющее из себя серверную часть и веб-страницу, позволяющее полноценно работать с RTKLIB, наблюдать за качеством решения, изменять настройки и работать с логами данных.
}

\keywordsen{
  software complex,
  web-interface,
  RTKLIB,
  high-precision positioning
}

\abstractcontenten{
The aim of this thesis is to create a new interface for RTKLIB - an open source program package for high-precision positioning. New interface will allow watching RTKLIB status and change its settings via a webpage.

This work discusses some problems of the current RTKLIB interface, as they have a negative effect on the project's applications. Also, the thesis overviews several examples of web-interfaces, used in devices, that have no other means of user interaction. Using the collected data, an interface prototype, which allows monitoring and interacting with RTKLIB through a webpage, was created. To implement the prototype, several programming modules, allowing to automate working with RTKLIB and transfer data to a webpage for visualization were designed.

Result of this thesis is a new application, that consists of a server side and a client side. It allows monitoring RTKLIB solution status, changing processing settings and working with data logs.
}
