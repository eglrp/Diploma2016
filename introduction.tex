%%%%%%%%%%%%%%%%%%%%%%%%%%%%%%%%%%%%%%%%%%%%%%%%%%%%%%%%%%%%%%%%%%%%%%%%%%%%%%%%
\intro
%%%%%%%%%%%%%%%%%%%%%%%%%%%%%%%%%%%%%%%%%%%%%%%%%%%%%%%%%%%%%%%%%%%%%%%%%%%%%%%%

Современному человеку сложно представить жизнь без спутниковой навигации. Эта технология очень прочно вошла в нашу повседневную жизнь, и пользуются ей миллионы. Сегодня, GPS, GLONASS и другие системы позволяют легко определить, например, улицу и дом, на которой находится человек. Точность же современного приемника средней ценовой категории варьируется от трех до пяти метров, в зависимости от условий. Для бытового применения, например ориентации в городе, это отличный результат. Человек вполне может сориентироваться при таких данных. Однако, спутниковые системы навигации имеют множество других применений, таких как автоматизация сельскохозяйственных работ или картография. Здесь нужно принципиально другое качество решения. Для таких применений используют технологию дифференциального GPS. Данное решение подразумевает использование сложных алгоритмов и продукты, доступные на рынке стоят дорого. Однако, существует RTKLIB - проект с открытым исходным кодом, реализующий эти самые алгоритмы для стандартных, общедоступных приемников. Распространению RTKLIB мешают неудобства использования: для управления и мониторинга требуется наличие полноценного компьютера, а программы пакета имеют множество режимов работы и настроек. Общий порог вхождения очень высок.


\textbf{Целью} данной работы является создание системы, позволяющей взаимодействовать с RTKLIB через браузер. Под взаимодействием понимается возможность наблюдать статус системы, возможность изменять настройки программы, а также работать с накопленными логами данных ГНСС.

Использование разрабатываемого программного комплекса позволит значительно упростить работу с RTKLIB, сделав акцент на важных настройках и функциональности, а самое главное, позволив работать с программами через браузер. Возможность работы через браузер важна по двум причинам. Во-первых, это означает возможность запуска RTKLIB на встраиваемых модулях без органов управления. Во-вторых, браузер присутствует на практически любых современных платформах и решение в виде веб-страницы будет совместимо с огромным количеством устройств. Это позволит использовать смартфон или планшет для наблюдения за работой RTKLIB, изменения настроек, скачивания логов данных.

Для достижения поставленной цели необходимо решение следующих \textbf{задач}:
\begin{enumerate}
  \item Проведение анализа возможностей RTKLIB;
  \item Анализ примеров использования веб-интерфейсов;
  \item Проектирование и разработка приложения;
  \item Тестирование приложения;
\end{enumerate}

\textbf{Результатом} данной работы стало полнофункциональное приложение, выполнящие следующие функции:
\begin{itemize}
  \item Переключение между различными ролями в RTK системе;
  \item Возможность изменять настройки основных режимов работы;
  \item Работа с конфигурационными файлами(создание, копирование, удаление);
  \item Отображение статуса системы - текущие координаты, качество решения;
  \item График уровней приема сигнала спутников;
  \item Cписок логов ГНСС данных с возможностью их скачать;
  \item Возможность сконвертировать логи ГНСС данных в формат RINEX;
\end{itemize}

