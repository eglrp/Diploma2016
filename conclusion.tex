\conclusion

В ходе работы был проведен анализ программного пакета RTKLIB, его возможностей и способов применения. В результате анализа была определена необходимость разработки нового интерфейса, основанного на веб-технологиях и позволившего бы использовать RTKLIB в новых условиях.

На начальном этапе были проанализированы функции, выполняемые разными программами пакета, и выделены основные из них. Также, были рассмотрены примеры веб-интерфейсов различных устройств, для которых веб-страница является основным способом взаимодействия с пользователем. На основе полученной информации были определены функциональные требования для интерфейса, и был создан прототип, который удовлетворяет новым требованиям.

Следующим этапом проектирования стало определение метода взаимодействия с самими программами пакета RTKLIB. Был выбран подход, при котором автоматизируется работа с уже скомпилированными программами. Также, было решено создать одностраничное веб-приложение, обменивающееся WebSocket сообщениями при работе с серверной частью. Учитывая архитектуру проекта, были выбраны основные инструменты разработки - язык Python, библиотека Pexpect и фрэймворк Flask для серверной части и jQuery и jQuery Mobile для стороны клиента.

В результате, было разработано полноценное приложение, позволяющее управлять приложениями RTKLIB посредством элементов веб-страницы. Приложение позволяет наблюдать за статусом обработки данных, видеть уровни приема спутников на специальном графике, изменять режимы работы и их настройки, а также работать с логами ГНСС данных.

На заключительном этапе разработки было проведено функциональное тестирование по разработанной программе и методике испытаний, которое подтвердило возможности нового интерфейса, такие как возможность запуска RTKLIB на вычислительном модуле без органов управления и использование любого устройства, например смартфона, для наблюдения за работой и конфигурацией ПО.

В качестве дальнейшего развития проекта, можно обратить внимание на две вещи. Во-первых, в качестве базы для интерфейса следует выбрать более современный и богатый набор элементов веб-страницы. Во-вторых, рассмотреть возможность создания нативных приложений для распространенных мобильных операционных систем, таких как iOS и Android. Это позволит снять задачу поиска нужного IP адреса в локальной сети с пользователя и снимет зависимость от кэша браузера.